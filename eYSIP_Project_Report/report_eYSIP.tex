\documentclass[a4paper,12pt,oneside]{book}

%-------------------------------Start of the Preable------------------------------------------------
\usepackage[english]{babel}
\usepackage{blindtext}
%packagr for hyperlinks
\usepackage{hyperref}
\hypersetup{
    colorlinks=true,
    linkcolor=blue,
    filecolor=magenta,      
    urlcolor=cyan,
}

\urlstyle{same}
%use of package fancy header
\usepackage{fancyhdr}
\setlength\headheight{26pt}
\fancyhf{}
%\rhead{\includegraphics[width=1cm]{logo}}
\lhead{\rightmark}
\rhead{\includegraphics[width=1cm]{logo}}
\fancyfoot[RE, RO]{\thepage}
\fancyfoot[CE, CO]{\href{http://www.e-yantra.org}{www.e-yantra.org}}

\pagestyle{fancy}

%use of package for section title formatting
\usepackage{titlesec}
\titleformat{\chapter}
  {\Large\bfseries} % format
  {}                % label
  {0pt}             % sep
  {\huge}           % before-code
 
%use of package tcolorbox for colorful textbox
\usepackage[most]{tcolorbox}
\tcbset{colback=cyan!5!white,colframe=cyan!75!black,halign title = flush center}

\newtcolorbox{mybox}[1]{colback=cyan!5!white,
colframe=cyan!75!black,fonttitle=\bfseries,
title=\textbf{\Large{#1}}}

%use of package marginnote for notes in margin
\usepackage{marginnote}

%use of packgage watermark for pages
%\usepackage{draftwatermark}
%\SetWatermarkText{\includegraphics{logo}}
\usepackage[scale=2,opacity=0.1,angle=0]{background}
\backgroundsetup{
contents={\includegraphics{logo}}
}

%use of newcommand for keywords color
\usepackage{xcolor}
\newcommand{\keyword}[1]{\textcolor{red}{\textbf{#1}}}

%package for inserting pictures
\usepackage{graphicx}

%package for highlighting
\usepackage{color,soul}

%new command for table
\newcommand{\head}[1]{\textnormal{\textbf{#1}}}


%----------------------End of the Preamble---------------------------------------


\begin{document}

%---------------------Title Page------------------------------------------------
\begin{titlepage}
\raggedright
{\Large eYSIP2017\\[1cm]}
{\Huge\scshape Title of the Project \\[.1in]}
\vfill
\begin{flushright}
{\large Intern 1 Name \\}
{\large Intern 2 Name \\}
{\large Mentor Name \\}
{\large Duration of Internship: $ 22/05/2017-07/07/2017 $ \\}
\end{flushright}

{\itshape 2017, e-Yantra Publication}
\end{titlepage}
%-------------------------------------------------------------------------------

\chapter[Project Tag]{Project Name}
\section*{Abstract}
Give the brief introduction and overview of the project

\subsection*{Completion status}
Give details for work/project completed successfully. If work is not
complete, mention the details till which task is done.

\section{Hardware parts}
\begin{itemize}
  \item List of hardware 
  \item Detail of each hardware: \href[page=5]{./datasheet/MPU-9150.pdf}{Datasheet, page 5}, \href{http://www.amazon.in}{Vendor link}, 
  \item Connection diagram
\end{itemize}

\section{Software used}
\begin{itemize}
  \item List of software used 
  \item Detail of software: version, \href{http://www.amazon.com}{download link}, 
  \item Installation steps
\end{itemize}

\section{Assembly of hardware}
Circuit diagram and Steps of assembly of hardware with pictures for each step
\subsection*{Circuit Diagram}
Circuit schematic, simplified circuit diagram , block diagram of system
\subsection*{Step 1}
Steps for assembling part 1
\subsection*{Step 2}
Steps for assembling part 2
\subsection*{Step 3}
Steps for assembling part 3



\section{Software and Code}
Github Repository with code and documentation: \href{https://github.com/eYSIP-2018/IFTTT-and-NLP}{Link}
\newline\\
\textbf{Webhook(fulfillment) program:}
\newline JavaScript program that recieves user query data from DialogFlow engine, calls corresponding rest APIs of iot-platform and returns the response to DialogFlow.
\newline\\
\textbf{natural-cron readableToCron.js:}
\newline JavaScript Library to convert normal English phrases into "Cron expressions". Visit \href{https://github.com/darkeyedevelopers/natural-cron.js}{here} for more details.

\section{Use and Demo}
Final Setup Image

\begin{enumerate}
\item Connect the hardware device to power supply and make sure it is switched on.
\item Verify that device has active internet connection.
\item Make sure all the servers are started and working as explained in installation guide.
\item Use the assistant(Web assistant on iot-platform, Google actions simulator, DialogFlow simulator or Google assistant on handheld devie) to fire queries.
\end{enumerate}

\section{Future Work}
What can be done to take this work ahead in future as projects.

\section{Bug report and Challenges}
\textbf{Efficiency of Assistant:}
\\Though efficiency of assistant has become quite good over the course of development, still the efficiency can be improved with more training.
\\\newline\textbf{Efficiency of natural-cron.js:}
\\Building a program that will convert english phrases to desired cron expressions was a major challenge. Efficiency of natural-cron.js library can be consistantly improved over the course of time to handle complex phrases.
\\\newline\textbf{Challenges:}
\begin{itemize}
\item Implementing OAuth2 token based authentication for login via Google Assistant
\item Making the assistant dynamic to handle all devices and sensors
\item Improving accuracy of dialogflow agent
\item Implementing save and restore for blockly blocks
\end{itemize}

\begin{thebibliography}{li}
[1] Justin Jose,
{\em Building Chatbots - A comparison of Rasa-NLU and Dialogflow},
\href{https://www.linkedin.com/pulse/building-chatbots-comparison-rasa-nlu-dialogflow-justin-jose/}{https://www.linkedin.com/pulse/building-chatbots-comparison-rasa-nlu-dialogflow-justin-jose/}
\\\newline
[2] DialogFlow,
{\em Reference documentation for Google's DialogFlow engine},
\href{https://dialogflow.com/docs/getting-started}{https://dialogflow.com/docs/getting-started}
\\\newline
[3] Blockly,
{\em Reference documentation for Google's Blockly library},
\href{https://developers.google.com/blockly/reference/overview}{https://developers.google.com/blockly/reference/overview}
\end{thebibliography}


\end{document}

