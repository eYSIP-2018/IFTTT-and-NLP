\documentclass[a4paper,12pt,oneside]{book}

%-------------------------------Start of the Preable------------------------------------------------
\usepackage[english]{babel}
\usepackage{blindtext}
%packagr for hyperlinks
\usepackage{hyperref}
\hypersetup{
    colorlinks=true,
    linkcolor=blue,
    filecolor=magenta,
    urlcolor=cyan,
}

\urlstyle{same}
%use of package fancy header
\usepackage{fancyhdr}
\setlength\headheight{26pt}
\fancyhf{}
%\rhead{\includegraphics[width=1cm]{logo}}
\lhead{\rightmark}
\rhead{\includegraphics[width=1cm]{logo}}
\fancyfoot[RE, RO]{\thepage}
\fancyfoot[CE, CO]{\href{http://www.e-yantra.org}{www.e-yantra.org}}

\pagestyle{fancy}

%use of package for section title formatting
\usepackage{titlesec}
\titleformat{\chapter}
  {\Large\bfseries} % format
  {}                % label
  {0pt}             % sep
  {\huge}           % before-code

%use of package tcolorbox for colorful textbox
\usepackage[most]{tcolorbox}
\tcbset{colback=cyan!5!white,colframe=cyan!75!black,halign title = flush center}

\newtcolorbox{mybox}[1]{colback=cyan!5!white,
colframe=cyan!75!black,fonttitle=\bfseries,
title=\textbf{\Large{#1}}}

%use of package marginnote for notes in margin
\usepackage{marginnote}

%use of packgage watermark for pages
%\usepackage{draftwatermark}
%\SetWatermarkText{\includegraphics{logo}}
\usepackage[scale=2,opacity=0.1,angle=0]{background}
\backgroundsetup{
contents={\includegraphics{logo}}
}

%use of newcommand for keywords color
\usepackage{xcolor}
\newcommand{\keyword}[1]{\textcolor{red}{\textbf{#1}}}

%package for inserting pictures
\usepackage{graphicx}

%package for highlighting
\usepackage{color,soul}

%new command for table
\newcommand{\head}[1]{\textnormal{\textbf{#1}}}

\usepackage{listings}
\usepackage{color}

\definecolor{dkgreen}{rgb}{0,0.6,0}
\definecolor{gray}{rgb}{0.5,0.5,0.5}
\definecolor{mauve}{rgb}{0.58,0,0.82}

\lstset{frame=tb,
  language=Java,
  aboveskip=3mm,
  belowskip=3mm,
  showstringspaces=false,
  columns=flexible,
  basicstyle={\small\ttfamily},
  numbers=none,
  numberstyle=\tiny\color{gray},
  keywordstyle=\color{blue},
  commentstyle=\color{dkgreen},
  stringstyle=\color{mauve},
  breaklines=true,
  breakatwhitespace=true,
  tabsize=3
}
%----------------------End of the Preamble---------------------------------------


\begin{document}

%---------------------Title Page------------------------------------------------
\begin{titlepage}
\raggedright
{\Large eYSIP2018\\[1cm]}
{\Huge\scshape Eyantra IoT Smart Assistant \& IFTTT for IoT \\[.1in]}
\vfill
\begin{flushright}
{\large Onkar J. Sathe \\}
{\large Rohit G. Rathi \\}
{\large Mentors : Omkar Manjrekar, Vikrant Fernandes, Deepa Avudiappan \\}
{\large Duration of Internship: $ 21/05/2018-06/07/2018 $ \\}
\end{flushright}

{\itshape 2018, e-Yantra Publication}
\end{titlepage}
%-------------------------------------------------------------------------------

\chapter[Project Tag]{eYantra IoT Smart Assistant \& IFTTT for IoT}
\section*{Abstract}
A web or an app interface for IoT may not always
be easy for people to work with, especially if they
aren’t used to such kinds of applications.
But what if you could tell someone to do things
on your behalf?
eYantra IoT Smart Assistant ( eYISA ) is inspired
from the goal to bring the magic of IoT to the
fingertips of users and accessible over voice
commands.
In AWS IoT writing rules is hard and not
that much intuitive you also need to know
SQL (for AWS IoT Rule Engine).
And it’s di
fficult to select devices over which you
want to write rules. A simple GUI needed to write
rules to invoke the actions on devices.

\subsection*{Completion status}
The project is successfully completed, all deliverables are achieved. We have successfully created Smart Assistant using dialogflow engine which able to perform all the frequent
queries that take place on IoT platform. \\
And also we have devloped flexible block based GUI for IFTTT rules.

\section{Hardware parts}
\begin{itemize}
  \item ESP DHT sensor
  \item 5 day workshop kit
  \item For connection diagram and setup refer this repository \href{https://github.com/sanamshakya/interfacing-AWS-IoT/}{Interfacing with AWS-IoT}
\end{itemize}

\section{Software used}
\begin{itemize}
  \item \href{dialogflow.com}{Dialogflow}
  \item \href{https://mongoose-os.com/}{Mongoose OS}
  \item \href{https://pagekite.net/}{PageKite} or \href{https://ngrok.com/}{ngrok}
  \item \href{https://cloud.google.com/functions/docs/emulator}{Cloud Functions Local Emulator}
  \item \href{https://developers.google.com/blockly/}{blockly}
\end{itemize}

\section{Assembly of hardware}
For circuit diagram and steps of assembly of hardware check this repository  \href{https://github.com/sanamshakya/interfacing-AWS-IoT/}{Interfacing with AWS-IoT}
\subsection*{Setup Guide}
To start follow this instruction of to install iot-platform and setup for AWS-IoT
\href{https://github.com/eYSIP-2018/IFTTT-and-NLP/wiki/Installation-Guide#setting-up-the-iot-platform}{Installation Guide}
\subsection*{Step 1}
After completing setup of \href{https://github.com/eYSIP-2018/IFTTT-and-NLP/tree/master/iot-platform}{iot-platform} and AWS-IoT then run server of iot-platform
\subsection*{Step 2 - For Development Setup }
Configure your PageKite account (its free of one month) then follow this \href{https://pagekite.net/support/quickstart/}{quickstart}.\\
\\
1. Tunnel iot-platform server to public ip using this command
\begin{lstlisting}
pagekite.py 8002 iotplatform-eyantra.pagekite.me
\end{lstlisting}
2. To deploy \href{https://github.com/eYSIP-2018/IFTTT-and-NLP/tree/master/eYantraWebhook}{webhook function} for fullfillment of dialogflow in Cloud Functions Local Emulator follow this \href{https://cloud.google.com/functions/docs/emulator}{guide} \\
3. Tunnel webhook function deployed in Cloud Functions Local Emulator to public ip using this command
\begin{lstlisting}
pagekite.py 8010 webhook-eyantra.pagekite.me
\end{lstlisting}
4. To setup Dialogflow follow this \href{https://github.com/eYSIP-2018/IFTTT-and-NLP/wiki/Installation-Guide#setting-up-the-assistant-dialogflow--webhook}{guide} upto 4th instruction.\\
5. Then after configuring all setting of dialogflow you will get Google Project Id in setting of your Agent.\\
6. Then enter the fullfillment URL as
\begin{lstlisting}
https://webhook-eyantra.pagekite.me/<YOUR_PROJECT_ID>/<YOUR_REGION>/eYantraWebhook
\end{lstlisting}
\subsection*{Step 2 - For Production Setup }
1. It is assumed that iot-platform server is hosted on cloud server like EC2(Amazon).\\
2.To deploy \href{https://github.com/eYSIP-2018/IFTTT-and-NLP/tree/master/eYantraWebhook}{webhook function} for fullfillment of dialogflow in \href{https://cloud.google.com/functions/}{Google Cloud Functions} follow this \href{https://cloud.google.com/functions/docs/quickstart-console}{guide}
3. Then enter the fullfillment URL as
\begin{lstlisting}
https://<YOUR_REGION>-<YOUR_PROJECT_ID>.cloudfunctions.net/eYantraWebhook
\end{lstlisting}
\section{Software and Code}
Github Repository with code and documentation: \href{https://github.com/eYSIP-2018/IFTTT-and-NLP}{Link}
\newline\\
\textbf{Webhook(fulfillment) program:}
\newline JavaScript program that recieves user query data from DialogFlow engine, calls corresponding rest APIs of iot-platform and returns the response to DialogFlow.
\newline\\
\textbf{natural-cron readableToCron.js:}
\newline JavaScript Library to convert normal English phrases into "Cron expressions". Visit \href{https://github.com/darkeyedevelopers/natural-cron.js}{here} for more details.

\section{Use and Demo}
\begin{enumerate}
\item Connect the hardware device to power supply and make sure it is switched on.
\item Verify that device has active internet connection.
\item Make sure all the servers are started and working as explained in installation guide.
\item Use the assistant(Web assistant on iot-platform, Google actions simulator, DialogFlow simulator or Google assistant on handheld devie) to fire queries.
\end{enumerate}

\section{Future Work}
\begin{itemize}
\item After increasing the e
fficiency of the assistant
 through more training, especially working
 with Hindi phrases then it can be used anywhere
\item Assistants can be made dynamic enough
 to adapt to other APIs, So APIs are accessible over
 Natural Language of users even in form of text
 or voice.
\end{itemize}
\section{Bug report and Challenges}
\textbf{Efficiency of Assistant:}
\\Though efficiency of assistant has become quite good over the course of development, still the efficiency can be improved with more training.
\\\newline\textbf{Efficiency of natural-cron.js:}
\\Building a program that will convert english phrases to desired cron expressions was a major challenge. Efficiency of natural-cron.js library can be consistantly improved over the course of time to handle complex phrases.
\\\newline\textbf{Challenges:}
\begin{itemize}
\item Implementing OAuth2 token based authentication for login via Google Assistant
\item Making the assistant dynamic to handle all devices and sensors
\item Improving accuracy of dialogflow agent
\item Implementing save and restore for blockly blocks
\end{itemize}

\begin{thebibliography}{li}
[1] Justin Jose,
{\em Building Chatbots - A comparison of Rasa-NLU and Dialogflow},
\href{https://www.linkedin.com/pulse/building-chatbots-comparison-rasa-nlu-dialogflow-justin-jose/}{https://www.linkedin.com/pulse/building-chatbots-comparison-rasa-nlu-dialogflow-justin-jose/}
\\\newline
[2] DialogFlow,
{\em Reference documentation for Google's DialogFlow engine},
\href{https://dialogflow.com/docs/getting-started}{https://dialogflow.com/docs/getting-started}
\\\newline
[3] Blockly,
{\em Reference documentation for Google's Blockly library},
\href{https://developers.google.com/blockly/reference/overview}{https://developers.google.com/blockly/reference/overview}
\end{thebibliography}


\end{document}
